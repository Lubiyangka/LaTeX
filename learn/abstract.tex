\begin{abstract}

    本文是《实践论》的读书报告,
    《实践论》是毛泽东主席在1937年2月全民族抗战统一战线建立之后的同年7月与延安抗日军政大学发表的。
    该时期党内部存在经验主义和教条主义,
    针对这种现象,
    毛主席写下了这篇文章,
    用马克思主义的观点去揭露了这种现象,
    为该时期的中国共产党,
    和中国人民指明了全民族抗战的新方向。

    本书用马克思主义认识论的观点,
    通过分析认知活动的过程,
    解释了什么是实践,
    指出实践的过程是实践,认识,再实践,再认识。
    如果把时间轴分为三部分:过去,现在和未来。
    身处现在的人,对于应该采取何种路线,即发展的方向,大致有两种:
    一是基于过去的认知,极端情况在历史上表现为右倾机会主义
    ;二是基于未来的认知,即是超过现在的实践,极端情况在历史上表现为左倾冒险主义。

    因此,本书指出人的认知运动是随着时代发展而发展的,
    但是人的认知是需要从感觉阶段到理性阶段,形成思想或理论;
    从思想或理论经过实践验证,变成真理或事实的,对于时代的发展是具有一定滞后性的,
    所以思想落后于实际的事事常有的。
    我们小组认为,这里的实践,认知,再实践,再认知,就是解决认知滞后性的方法,把实践认知的过程变成的动态过程,以适应急剧变化的时代

\end{abstract}