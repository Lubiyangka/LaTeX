% 文档类
\documentclass[a4paper, UTF8, onecolumn]{ctexart}
% 导入宏包
\usepackage{makecell, tabularx, multirow}

% 编译但是不生成pdf
\usepackage{syntonly}
% \syntaxonly

\includeonly{abstract}


% 导言部分
\begin{document}


% 另起一页插入
\begin{abstract}

    本文是《实践论》的读书报告,
    《实践论》是毛泽东主席在1937年2月全民族抗战统一战线建立之后的同年7月与延安抗日军政大学发表的。
    该时期党内部存在经验主义和教条主义,
    针对这种现象,
    毛主席写下了这篇文章,
    用马克思主义的观点去揭露了这种现象,
    为该时期的中国共产党,
    和中国人民指明了全民族抗战的新方向。

    本书用马克思主义认识论的观点,
    通过分析认知活动的过程,
    解释了什么是实践,
    指出实践的过程是实践,认识,再实践,再认识。
    如果把时间轴分为三部分:过去,现在和未来。
    身处现在的人,对于应该采取何种路线,即发展的方向,大致有两种:
    一是基于过去的认知,极端情况在历史上表现为右倾机会主义
    ;二是基于未来的认知,即是超过现在的实践,极端情况在历史上表现为左倾冒险主义。

    因此,本书指出人的认知运动是随着时代发展而发展的,
    但是人的认知是需要从感觉阶段到理性阶段,形成思想或理论;
    从思想或理论经过实践验证,变成真理或事实的,对于时代的发展是具有一定滞后性的,
    所以思想落后于实际的事事常有的。
    我们小组认为,这里的实践,认知,再实践,再认知,就是解决认知滞后性的方法,把实践认知的过程变成的动态过程,以适应急剧变化的时代

\end{abstract}
% 在文中插入
% \begin{abstract}

    本文是《实践论》的读书报告,
    《实践论》是毛泽东主席在1937年2月全民族抗战统一战线建立之后的同年7月与延安抗日军政大学发表的。
    该时期党内部存在经验主义和教条主义,
    针对这种现象,
    毛主席写下了这篇文章,
    用马克思主义的观点去揭露了这种现象,
    为该时期的中国共产党,
    和中国人民指明了全民族抗战的新方向。

    本书用马克思主义认识论的观点,
    通过分析认知活动的过程,
    解释了什么是实践,
    指出实践的过程是实践,认识,再实践,再认识。
    如果把时间轴分为三部分:过去,现在和未来。
    身处现在的人,对于应该采取何种路线,即发展的方向,大致有两种:
    一是基于过去的认知,极端情况在历史上表现为右倾机会主义
    ;二是基于未来的认知,即是超过现在的实践,极端情况在历史上表现为左倾冒险主义。

    因此,本书指出人的认知运动是随着时代发展而发展的,
    但是人的认知是需要从感觉阶段到理性阶段,形成思想或理论;
    从思想或理论经过实践验证,变成真理或事实的,对于时代的发展是具有一定滞后性的,
    所以思想落后于实际的事事常有的。
    我们小组认为,这里的实践,认知,再实践,再认知,就是解决认知滞后性的方法,把实践认知的过程变成的动态过程,以适应急剧变化的时代

\end{abstract}

% 正文部分
``Hello world!'' from \LaTeX

this is \TeX no, but \TeX{} yes

换行\par 特殊字符\# \% \textbackslash

连词现象比较常见It is diffcult to find \ldots \\
but in this way, it is not dif{}f{}cult to f{}ind \ldots \\
`单引号' ``双引号''
连字号 X-ray
短破折号 1--100
长破折号 yes---no\\
省略号 \ldots
波浪号 \~\\

H\^otel, na\"\i ve, \'el\`eve,\\
 sm\o rrebr\o d, !`Se\ norita!,\\
 Sch\"onbrunner Schlo\ss{}
 Stra\ss e
\\

国际上,1929年经济危机之后,世界主要经济体都受到了极大的冲击,为此不同的国家出现了化解经济危机的方案。
德意日的法西斯主义逐渐崛起,将经济危机的矛盾转化为民族矛盾,进而对外发动侵略战争;
美国为首的资本主义国家选择国家资本主义,用“以工代赈”等一系列方式;
以苏联为首的社会主义国家,则通过社会主义的优势受到较小的冲击。
\\

国内上,西安事变之后,经过中国共产党人和中国国民党有志之士的共同努力,
在中共中央1937年2月10日致国民党五届三中全会的五项要求基础上,进一步谈判与协商,从事实上达成了抗日民族统一战线。

当时,党内出现的部分教条主义者长期拒绝中国革命的经验,
曲解马克思主义主义书籍的以言片语,和部分经验主义者长期只基于自己的经验,
盲目工作。针对这两种错误,毛泽东主席在抗大的指导学习上,写出了这篇文章。

本文从认知活动的过程为索引,以马克思主义为理论基础,展开论证了实践与认识的现象与本质。

首先,从马克思主义者的角度出发,指出社会实践是认识的唯一来源,生产活动是最基本的活动。
人们的生产实践及形成的生产关系,决定了人类社会其他各种关系。
社会的历史是阶级斗争的历史,因此阶级是固有的,每个人都在一定的阶级中生活,这是马克思主义唯物论与离开人的社会性和历史发展来观察问题的唯物论的根本区别。

社会实践是认识发展的动力。
人类社会的生产活动不断由低级向高级发展,这决定和推动了人类的认识不断由低级向高级、由片面性向科学性发展。
马克思、恩格斯生活在资本主义社会化大生产的时代,吸收了人类社会创造的全部先进知识财富,又冲破了剥削阶级的偏见,科学地认识和把握了人类社会发展的客观规律。

社会实践是检验认识真理性的唯一标准。
认识只有在社会实践中取得了预想结果,才能被证实。
实践中出现失败或错误,则说明主观与客观是不一致的,需要进一步改进认识。
辩证唯物论把实践置于第一位,不仅因为认识来源于实践,更是因为实践是检验认识真理性的唯一标准。

要坚持主观和客观、理论和实践具体的历史的统一
。革命队伍中的顽固派、右倾机会主义和“左”翼空谈主义,唯心论和机械唯物论,机会主义和冒险主义,都是以主观和客观相分裂,以认识和实践相脱离为特征的。
坚持不断地使自己的认识符合客观情况的变化,在实践中不断调整和改变不符合实际的认识,不断保持认识与实践的一致性,才能防止脱离实际。

读完《实践论》之后,我们小组发现除了毛主席在文中举的无产阶级对资本主义社会的认识,
中国人民对帝国主义的认识,对于战争的认识之外,我们的日常生活中也有许多示例可以去验证文中提到的观点。

比如,学习一门课程的过程就是一个认识活动:阅读书籍资料了解并学习该课程相关的知识点和内容就是一个认知的感性阶段,
基于前序课程的理解和个人阅历的分析,我们去尝试了解该课程的某些知识,即是了解“现象”;做一些课后习题的练习则是一个认知的理性阶段,
我们不再是从自己感性的角度出发,而是基于对于这门课程知识点的掌握与学习,去做题解题,即是我们对于“现象”的判断和推理,可以看作是认识活动的低级阶段。
如果说,只阅读不做习题,就会出现,感性认识高于理性认识;只做习题不阅读,就是理性认识高于感性认识。
但是如果我们可以在阅读的基础上做习题,在做习题之后再回过头阅读,就会基本上掌握这门课程的八成左右,这就是文中提到的“理性认识依赖于感性认识,感性认识有待于发展到理性认识”,这是一个实践,认识的过程。

如果说对于这门课的学习到这里就结束的话,应该是不能完全贯通的,因此我们需要进入认识活动的高级阶段,从解释客观事物的规律到通过这些客观规律去改造。
通俗的讲,就是我们学到这里已经可以应付题目了,这就是解释;
接下来要做的就是实践,去验证课本上的内容,去动手实践。我们发现大多数的工科课程都是这样的流程:课堂学习,测试练习,实验操作。这刚好与前面讲到的感性,理性,实践三阶段契合。

但是本书讲到的是实践,认识,再实践,再认识,到这里对于这门课的学习只能说做到了前两点。
而忽略了学科知识是不断发展的,是需要不断再学习,再理解,再实践的,如果做到了这一点,那么我们就迈入了该学科的学术层面,向着更深处探索。

以上的这段论述与理解,本质上算是我们小组在阅读原文的感性阶段之后做出的一次验证,这个过程本身也是实践与认识的过程,这更加表明了《实践论》对青年的现世意义。
这是对《实践论》学习的一次实践与认知,虽然本文到这里也就要结束了,但是我们在本次阅读中的所得和对于《实践论》理论的再实践与再认知将会伴随我们一生。


\end{document}

% "\end之后的部分会被忽略"